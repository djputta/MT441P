% !TEX program = xelatex

\documentclass[12pt, answers]{exam}

\usepackage{mathtools}
\usepackage[MnSymbol]{mathspec}
\usepackage{titling}
\usepackage{geometry}

\newgeometry{hmargin={12mm,17mm}}

\setmainfont[Numbers={Lining,Proportional}]{Minion Pro}
\setmathsfont(Digits,Latin,Greek)[Numbers={Lining,Proportional}]{Minion Pro}

\DeclareMathAlphabet{\mathbb}{U}{msb}{m}{n}

\pagestyle{headandfoot}
\runningheadrule
\footrule
\extraheadheight{-.3in}
\extrafootheight{-.8in}
\firstpageheader{}{}{}
\runningheader{\large\bfseries MT441P}{\large\bfseries Dheeraj Putta}{\large\bfseries Assignment 4}
\footer{}{Page \thepage\ of \numpages}{}

\setlength{\droptitle}{-5em}

\title{MT441P - Assignment 4}
\author{Dheeraj Putta \\ 15329966}
\date{}
\newcommand{\norm}[1]{\left\lVert#1\right\rVert}
\renewcommand{\d}{\mathrm{d}}

\linespread{1.15}

\begin{document}
    \maketitle
    \begin{questions}
        \thispagestyle{foot}
        \setcounter{question}{1}
        \question Let $|G| = 144$. Show that $G$ is not simple.
        \begin{solution}
            We know that $|G| = 144 = 2^43^2$.\\
            We have that $n_3 = 1, 4$ or $16$.
            \begin{itemize}
                \item If $n_3 = 1$ then $G$ is not simple since the $3$-Sylow subgroup is normal.
                \item Suppose $n_3 = 4$. Consider the following map:
                \[ \phi\colon G \to \text{Perm}(G/N_G(Q)) \]
                where $Q$ is a $3$-Sylow subgroup. We know that $\phi \cong S_4$ as $[G: N_G(Q)| = 4$. Since the order of $G$ is less than the order of $S_4$, we see that we have a homomorphism with a non-trivial kernel. Since $\ker \phi$ is nontrivial and normal in $G$, $G$ is not simple.

                \item Suppose $n_3 = 16$. Now suppose that every pair of 3-Sylow subgroups have trivial intersection. This means that there are $16 \cdot 8$ non-identity elements in the $3$-Sylow subgroups and there are 16 elements left to form the $2$-Sylow subgroup, which means that there is only one $2$-Sylow subgroup. Therefore $n_2 = 1\implies G$ is not simple. \\

                Suppose there exists two different Sylow 3-subgroups P and Q such that $|P \cap Q| = 3$. Let $N = N_G(P \cap Q)$ be the normalizer of $P \cap Q$, which is the largest subgroup of $G$ that contains $P \cap Q$ as a normal subgroup. Since $|P_i| = 3^2$, we know that $P_i$ is abelian for all $i$ and therefore $Q \cap P$ is normal in both $P$ and $Q$. So we have that both $P, Q \subset N \implies PQ \subset N$. We also have that
                \[ |PQ| = \frac{|P||Q|}{|P \cap Q|} = 27 \]
                Hence $|N| > 27$ as 27 does not divide the order of $G$. We also know that $9 \mathrel{|} |N|$. So we have that $|N| = 36, 72$ or $144$. If $|N| = 144$ then it would mean that $P \cap Q$ is normal in $G$. If $|N| = 36$ or $72$ then $[G \colon N] = 4$ or $2$. From a similar argument as above, we have $\phi\colon G \to \text{Perm}(G/N)$ to get that $\ker\phi$ is a nontrivial subgroup of $G$, hence $G$ is not simple.
            \end{itemize}
        \end{solution}
        \setcounter{question}{4}
        \question Suppose that $|G| = 60$ and that $G$ has 20 elements of order 3. Show that $G\cong A_5$.
        \begin{solution}
            Since there are 20 elements of order 3, it must be the case that the intersection of each of the subgroups is
            trivial, so we get that $n_3(3 - 1) = 20 \implies n_3 = 10$. \\ \\
            Now assume that $G$ is not simple and that it contains a non-trivial proper subgroup $H$.
            \begin{itemize}
                \item If $3 \mathrel{|} |H|$, then $H$ contains a $3$-Sylow subgroup $P$ of $G$. Since $H$ is normal in $G$
                      and every $3$-Sylow subgroup is a conjugate of $P$, $H$ contains all 10 3-Sylow subgroups
                      $\implies |H| \ge 21$. Since $|H| \mathrel{|} |G|$ and $H$ is proper, $|H| = 30$. This contradicts
                      the fact that every group of order 30 has a unique 3-Sylow subgroup. Therefore $3 \mathrel{\nmid} |H|$.
                \item If $|H| = 10$ or $20$, then $H$ has a normal Sylow subgroup which is also normal in $G$ (by Question 9).
                    Thus we may assume, by replacing $H$ by its normal subgroup if necessary, that $|H| = 2, 4$ or 5. Then
                    $|G / H| = 30, 15$ or 12. Which implies that $G / H$ has a normal subgroup $K / H$ of order 3, where
                    $K$ is a normal subgroup of $G$. Now $|K| = |K / H| \cdot |H| = 5 \cdot |H|$. This implies that
                    $5 \mathrel{|} |K|$ which contradicts the previous point. Therefore $G$ is simple.
            \end{itemize}
            Since we have already shown that a simple group of order 60 is isomorphic to $A_5$, we are finished.
        \end{solution}

        \question Let $G$ be the group of rotations of the Icosahedron. Show that $G\cong A_5$.
        \begin{solution}
            The elements of $G$ are:
            \begin{itemize}
                \item 4 rotations (by multiples of $2\pi / 4$) about centres of 6 pairs of opposite vertices.
                \item 1 rotation  (by $\pi$) about centres of 15 pairs of opposite edges.
                \item 2 rotations (by $\pm 2\pi / 3$) about 10 pairs of opposite faces.
            \end{itemize}
            Since $|G| = 60$ and the number of 3 cycles is 20 we can use the question above to show that $G \cong A_5$.
        \end{solution}

        \question Let $|G| = 3^3 \cdot 5 \cdot 7$. Show that $G$ is not simple.
        \begin{solution}
            By Sylow's Theorem $n_3$ must divide $35$ and $n_3$ has to be one more than a multiple of 3, thus
            $n_3 \in \{ 1, 7 \}$. Now suppose that $G$ is simple, then $n_3 \neq 1$. Then we have that $G$ is isomorphic
            to a subgroup of $S_7$ and that $|G|$ has to divide $|S_7|$. But this is false hence we have a contraction. Therefore
            $G$ is not simple.
        \end{solution}

        \question If $|G| = 2p$, $(p > 2)$ show that $G$ is cyclic or dihedral.
        \begin{solution}
            Choose $a, b \in G$ with $o(a) = 2$ and $o(b) = p$. Let $H = \langle a \rangle$ and $K = \langle b \rangle$.
            Since every subgroup of index 2 is normal, $K$ is normal in $G$. Thus $aba^{-1} = b^i$ for some integer $i$.
            We also know that $H \cap K = \{1\}$ and hence $G = HK = \langle a, b \rangle$. Now
            \[ b^{i^2} = (aba^{-1})^{i} = ab^ia^{-1} = a(aba^{-1})a^{-1} = b\]
            as $a^2 = 1$.\\
            Thus $b^{i^2 - 1} = 1$ and hence $p \mathrel{|} i^2 - 1$ as $o(b) = p$. Therefore either $p \mathrel{|} i - 1$
            or $p \mathrel{|} i + 1$.
            \begin{itemize}
                \item If $p \mathrel{|} i - 1$ then $aba^{-1} = b^i = b$ and so $ab = ba$. Thus $G$ is abelian and
                $o(ab) = 2p$, so $G$ is cyclic in this case.
                \item If $p \mathrel{|} i + 1$ then $aba^{-1} = b^i = b^{-1}$ and so
                \[ G = \langle a, b \colon a^2 = b^p = 1, aba^{-1} = b^{-1} \rangle \cong D_{2p} \]
            \end{itemize}
            Therefore $G$ is either cyclic or dihedral.
        \end{solution}

        \question Suppose that $N$ is a normal subgroup of $G$, and that $P$ is a $p$-Sylow subgroup of $N$.
        Show that $G = N_G(P)N.$
        \begin{solution}
            Pick $g \in G$. Since $P \subset N$ and $N \trianglelefteq G, gPg^{-1} \subset N$. Then by applying Sylows
            theorem to $N$, there is an $n \in N$ such that $gPg^{-1} = nPn^{-1}$, so $n^{-1}gPg^{-1}n = P$. That means
            $n^{-1}g \in N_G(P)$, so $g \in nN_G(P)$. Thus $G = N \cdot N_G(P)$.
        \end{solution}

        \question Suppose that $N \trianglelefteq G$, and that $P$ is a $p$-Sylow subgroup of $N$ and $P \trianglelefteq N$.
        Show that $P \trianglelefteq G$.
        \begin{solution}
            We know that $G = N_G(P)N$. However since $P$ is normal in $N$, we know that $N \leq N_G(P)$. This implies that
            $N_G(P)N = N_G(P)$. Therefore $G = N_G(P)$. Since the normalizer is the whole group, $P$ is normal in $G$.
        \end{solution}
        \pagebreak
        \question Let $P$ be a $p$-Sylow subgroup of $A_n, n\geq 3$. Show that $N_G(P)$ contains an odd permutation.
        \begin{solution}
            Suppose $G = S_n$.
            By Frattini's argument we get that $S_n = A_nN_G(P)$. Since $A_n$ contains only even permutations, the only
            way to obtain odd permutations is to compose it with odd permutations. Therefore $N_G(P)$ must contain an odd
            permutation.
        \end{solution}
    \end{questions}
\end{document}
