% !TEX program = xelatex

\documentclass[12pt, answers]{exam}

\usepackage{mathtools}
\usepackage[MnSymbol]{mathspec}
\usepackage{titling}
\usepackage{geometry}
\usepackage{nicefrac}
\usepackage{xfrac}

\newgeometry{hmargin={12mm,17mm}}

\setmainfont[Numbers={Lining,Proportional}]{Minion Pro}
\setmathsfont(Digits,Latin,Greek)[Numbers={Lining,Proportional}]{Minion Pro}

\DeclareMathAlphabet{\mathbb}{U}{msb}{m}{n}

\pagestyle{headandfoot}
\runningheadrule
\footrule
\extraheadheight{-.3in}
\extrafootheight{-.8in}
\firstpageheader{}{}{}
\runningheader{\large\bfseries MT441P}{\large\bfseries Dheeraj Putta}{\large\bfseries Assignment 5}
\footer{}{Page \thepage\ of \numpages}{}

\setlength{\droptitle}{-5em}

\title{MT441P - Assignment 5}
\author{Dheeraj Putta \\ 15329966}
\date{}
\newcommand{\norm}[1]{\left\lVert#1\right\rVert}
\renewcommand{\d}{\mathrm{d}}

\linespread{1.15}

\begin{document}
    \maketitle
    \begin{questions}
        \thispagestyle{foot}
        \question  How many distinguishable colourings of the regular tetrahedron are
        there, if you may use 5 colours?

        \begin{solution}
            We have that $|\Omega| = 5^4$.
            \begin{itemize}
                \item There are 8 rotations by $\sfrac{2\pi}{3}$ about an axis passing through a vertex and the centre of
                      the opposite face.
                \item There are 3 rotations by $\pi$ about an axis through 2 opposite edges.
            \end{itemize}
            Then we get that the number of distinct colourings is
            \[ \frac{1}{12}\left(5^4 + 11\left(5^2\right)\right) \]
        \end{solution}

        \question  How many distinguishable colourings of the cube are there, if you may
        use 10 colours?

        \begin{solution}
            We have that $|\Omega| = 10 ^6$.
            \begin{itemize}
                \item There are 6 rotations around an edge by $\pi$.
                \item There are 8 rotations around a vertex by $\pm \nicefrac{2\pi}{3}$.
                \item There are 6 rotations around a face by $\pm \nicefrac{\pi}{2}$.
                \item There are 3 rotations around a face by $\pi$.
            \end{itemize}
            Then we get that the number of distinct colourings is
            \[ \frac{1}{24}\left(10^6 + 12\left(10^3\right) + 11 \left(10^2\right)\right) \]
        \end{solution}

        \question How many distinguishable colourings of the octahedron are there, if you
        may use 8 colours?

        \begin{solution}
            We have that $|\Omega| = 8^8$.
            \begin{itemize}
                \item There are 6 rotations around a vertex by $\nicefrac{\pi}{2}$.
                \item There are 6 rotations around an edge by $\pi$.
                \item There are 3 rotations around a vertex by $\pi$.
                \item There are 8 rotations around a face by $\pm \nicefrac{2\pi}{3}$.
            \end{itemize}

            Then we get the number of distinct colourings is
            \[ \frac{1}{24} \left(8^8 + 6\left(8^2\right) + 17\left(8^4\right)\right) \]
        \end{solution}

        \question How many distinguishable colourings of the dodecahedron are there, if
        you may use 4 colours?

        \begin{solution}
            We have that $|\Omega| = 4^{12}$.
            \begin{itemize}
                \item 4 rotations (by multiples of $\nicefrac{2\pi}{4}$) about centres of 6 pairs of opposite faces.
                \item 1 rotation  (by $\pi$) about centres of 15 pairs of opposite edges.
                \item 2 rotations (by $\pm \nicefrac{2\pi}{3}$) about 10 pairs of opposite vertices.
            \end{itemize}
            Then we get that the number of distinct colourings is
            \[ \frac{1}{60}\left(4^{12} + 24\left(4^4\right) + 20\left(4^4\right) + 15\left(4^6\right)\right) \]
        \end{solution}

        \question How many distinguishable colourings of a roulette wheel are there, with
        20 identical slots, using 6 colours?

        \begin{solution}
            We know that for a roulette wheel with $n$ segments and $k$ colours the number of distinct colourings is:
            \[ \frac{1}{n}\sum_{\substack{d = 1 \\ d \mathrel{|} n}}^n \varphi(d)k^{\nicefrac{n}{d}}\]
            Then for $n = 20$ and $k = 6$, we get
            \begin{align*}
                \#\text{orbits } &= \frac{1}{20}\left( \varphi(1)6^{20} + \varphi(2)6^{10} + \varphi(4)6^5 + \varphi(5)6^4 +
                \varphi(10)6^2 + \varphi(20)6^1\right) \\
                &= \frac{1}{20}\left(6^{20} + 6^{10} + 2\left(6^{5}\right) + 4\left(6^4\right) + 4\left(6^2\right)
                    + 8\left(6^1\right)\right)
            \end{align*}
        \end{solution}

        \question  How many distinguishable necklaces can you make (with no clasp), using
        9 beads, where each can be any of 10 colours?

        \begin{solution}
            We know that for $n$ beads, $n$ odd, and $k$ colours the number of orbits is:
            \[ \frac{1}{2n}\left(\sum_{\substack{d = 1 \\ d \mathrel{|} n}}^n \varphi(d)k^{\nicefrac{n}{d}} + n\left(k^{\frac{n+1}{2}}\right)\right) \]
            Then for $n = 9$ and $k = 10$, we get
            \begin{align*}
                \#\text{orbits } &= \frac{1}{18}\left(\varphi(1)10^9 + \varphi(3)10^3 + \varphi(9)10^1 + 9\left(10^{5}\right)\right) \\
                &= \frac{1}{18}\left(10^9 + 2\left(10^3\right) + 6\left(10^1\right) + 9\left(10^{5}\right)\right)
            \end{align*}
        \end{solution}

        % \question  How many distinguishable necklaces can you make (with no clasp), using
        % $n$ beads, where each can be any of $k$ colours? Show that If $n$ is odd the
        % answer is
        % \[ \frac{1}{2n} \left[ \sum_{d \mathrel{|} n} \varphi(d)k^{\frac{n}{d}} + nk^{\frac{n+1}{2}}\right] \]
        % and if $n = 2m$ is even the answer is
        % \[ \frac{1}{2n} \left[ \sum_{d \mathrel{|} n} \varphi(d)k^{\frac{n}{d}} + mk^{m} + mk^{m+1}\right] \]

        % \begin{solution}
        %     If $n$ odd then we know that
        %     \begin{itemize}
        %         \item $|\text{Fix}(1)| = k^n$.
        %         \item Since necklaces act like a roulette wheel without flips we have that $|<\sigma^i>| = d$ where $d \mathrel{|} n$.
        %         We also know that there are $\varphi(d)$ of these types.
        %         \item For subgroups of the form $\tau\sigma^i$ these have order 2. We also know that there are $n$ of these
        %         subgroups.
        %     \end{itemize}
        %     Putting these together we get that
        %     \[ \#\text{orbits } &= \frac{1}{2n}\left[  \right] \]
        % \end{solution}
    \end{questions}
\end{document}
